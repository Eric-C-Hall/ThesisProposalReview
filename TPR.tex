\documentclass{article}

\title{Thesis Proposal Review}
\author{Eric Hall}

\begin{document}

\maketitle

\section{Thesis Proposal}

\subsection{Purpose of Thesis}

The thesis will be to produce a formally verified implementation of the Viterbi algorithm.

To create a principled, foundational implementation of error correcting code techniques. Existing implementations aren't as foundational, or ...

Whereas other approaches take an approach where they formally prove properties in the programming language itself, using interactive-theorem prover based methods have the additional improvements of increased fidelity of proof techniques, as you do not have to rely on an automated system to do all the proof for you. Furthermore, the automated system often takes a long time to perform its proofs. Looking at things from a human perspective and proving thnigs manually gives you the ability to direct the proof and use human knowledge explicitly to prove known results.

\subsection{What is the Viterbi algorithm}

The Viterbi algorithm is a 

\subsection{Important question: Why Viterbi and not something else? What advantages does Viterbi have?}

\subsection{Important question: What research has been done so far?}

\subsection{}

\subsection{Starshot Resilient multi-mission space}

This project aligns closely with defence innovation network StarShot priority areas.

In particular, it would be useful in ``secure and resilient communications delivered from space for a highly networked force.''

\subsection {Copy pasted from first research proposal}

The purpose of this project is to use an interactive theorem prover to formally prove foundational mathematical theorems in the area of algebra, including such areas as group theory, field theory, and Galois theory. Future researchers will be able to build upon these proofs, improving our confidence that our mathematical theories are correct.

Theorem proving has traditionally been done by hand, but this is time-consuming and prone to error. While humans are very good at understanding things on a conceptual level, they have a tendency to make silly mistakes when performing simple, boring, repetitive tasks, and the conceptual understanding they have of the topic has a tendency to obscure unjustified leaps in logic that appear “intuitive” to the human. Furthermore, the level of confidence that a third party has on the validity of a proof depends largely on how much they trust that the person performing the proof has done a good job at verifying their proof is correct. If they do not have such trust, then they must perform the boring, repetitive work of verifying the proof for themselves.

The simple, boring and repetitive nature of checking the validity of a proof makes it a prime target for automation by computers, which excel at simple and repetitive tasks. By using a computer to perform the brute work of verifying the correctness of the proof, the user is able to prove theorems to an impeccably high standard of formality, without the arduous process of checking that each stage of the proof is formally correct. It also allows a third party to have a high level of trust in the validity of the proven theorem without having to check the proof themselves, so long as they trust that the proof-checking software has been programmed correctly.

Furthermore, some theorem proving can be performed automatically by computers. This is best suited to simple and tedious derivations that need to be performed but do not require a high degree of understanding and are simply routine proofs. Using an automated theorem prover to make these derivations will take some of the load off the human prover, freeing the human from being bogged down in the minutiae of formal proof. Beyond that, the automated theorem provers may also have the ability to perform non-trivial proofs without the expenditure of human effort.

Automated theorem provers have weaknesses, though. The place where humans really shine is in having a deep understanding of the meaning behind the theorems. Whereas the computer is just manipulating symbols, the human will typically have a better conceptual understanding of what these symbols mean and why they are important, and how they relate to each other. While work is ongoing in order to improve the ability of computers to attain this kind of understanding, this is still one of their weaknesses and humans will have an advantage in solving complex problems where a general understanding of the problem to be solved is useful.

In this way, the human and computer solver cover each other’s weaknesses: the human excels at understanding the problem, whereas the automated theorem prover is very good at performing the grunt work.

One major application area of interactive theorem proving is in proving the correctness of vital software in scenarios in which no bugs are tolerable. Examples of such application areas include cryptography, vote-counting software, medical software, software for controlling spacecraft or other vehicles, software for controlling industrial machinery such as nuclear power plants, etc. In such scenarios where any bug could have a catastrophic effect, it is important to be confident that your systems work as intended. By using an interactive theorem prover to prove that the output of the relevant code is correct, we may achieve the necessary confidence that our code is safe.

Interactive theorem proving is also useful for improving our confidence in our understanding of mathematics and adjacent fields in general. Mathematics underpins much of the modern world, and to be confident in the effectiveness of many real-world systems, it is important to be confident in the underlying mathematics. The process of formalizing our mathematical theories is likely to expose flaws in our current understanding of mathematics, especially in the less well-trodden areas where the proofs have not been exposed to the level of scrutiny that is typical of more widely studied areas of mathematics.

My specific project aims to be a foundational building block in achieving these goals. I aim to produce a computer-assisted formalization of commonly used mathematical theorems, so that future researchers can build on top of this to produce computer-formalized proofs of theorems that are important in their own application areas. This could be used by future researchers both in domain-specific applications and in simply solidifying our understanding of mathematics in general. 

The specific area I would like to work in is algebra: in particular, areas such as group theory, field theory, and Galois theory. These are fundamental areas of study of vital importance to a large number of other disciplines, which is why I chose them as a topic of study.

As of right now, I am still in the beginning stages of literature review to identify pre-existing research on the topic. There does seem to be at least some relevant literature already on the topic of the formalization of Galois theory (see [1] and [2]). Although research has been done in this area before, there is still likely room to perform more research, as there are many useful theorems and only some will have been formalized. In addition, I expect that there are many closely related fields of algebra which have not yet had formalization work applied to them.

Novel ideas will be required in order to formalize the mathematics, because the theorems have not been proven to the required standard of rigor necessary. The existing proofs will likely be in a human-readable form rather than in a formal, computer-readable form, and as a result they will likely be missing key details necessary to make the proofs work in an interactive theorem prover setting.

References:

[1]    T. Browning and P. Lutz. “Formalizing Galois Theory”. Experimental Mathematics, vol. 31, pp 413-424, 2021

[2]    N. Curiel. “Formalizing Galois Theory I: Automorphism Groups of Fields”. Master’s Thesis, California State University San Marcos, MSc, 2011

[3]    M. Norrish. “Mechanised Computability Theory”. International Conference on Interactive Theorem Proving, pp 297-311, 2011

[4]    R. Affeldt, J. Garrigue and T. Saikawa. “Formalization of Reed-Solomon codes and progress report on formalization of LDPC codes”. International Symposium on Information Theory and Its Applications, pp 532-536, 2016

https://ieeexplore.ieee.org/document/7840481 Formalization of Reed-Solomon codes and progress report on formalization of LDPC codes

\subsection{Copy pasted from second research proposal}

The purpose of this project is to use an interactive theorem prover to produce a formally verified implementation of the Viterbi algorithm. This would also involve the formalization of underlying mathematical theorems which could be of general use.

When reading data from a noisy channel, errors are likely to occur. If we provide redundant information over this channel, then when an error occurs, it is possible to use the redundant information to correct the error with high probability [1]. Encodings of data that allow for this error-correcting property are called error-correcting codes.

The Viterbi algorithm is an algorithm that is used to decode a certain class of error-correcting codes [2].

Interactive theorem proving is a technique used to formally prove that a system works correctly with respect to a specification [3]. As long as there is trust that the theorem prover and the specification are both correct, this provides an 100% guarantee that the algorithm behaves correctly. This is useful in situations where failure could be catastrophic and thus we need extremely reliable software, for example in pacemakers or self-driving cars.

It would be useful to have a formally verified implementation of the Viterbi algorithm, because the algorithm may be used in systems that have a requirement of high reliability. It is extremely common to need to read information from a potentially noisy channel. Common examples include reading from a wireless connection [4] or even from a hard drive [5]. Most real-world systems would have a need to do these kinds of basic tasks, thus, most real-world systems have a need for error-correcting codes to some degree. This includes those systems that would benefit from the correctness guarantees provided by formal proof. Thus, the improved reliability of having a formally verified implementation of the Viterbi algorithm would be useful in practice in real-world systems with low fault tolerance.

Formalization of other error correcting codes has been performed in the past. For example, the correctness of Reed-Solomon codes has been formally proved using interactive theorem provers [6]. Also, the correctness of the Viterbi algorithm has been proven using handwritten methods [2], although handwritten methods are by nature less reliable than computer-verified methods. However, in my preliminary review of the literature, I have not encountered a computer verified formalization of the Viterbi algorithm. Thus, it should satisfy the novelty requirements of a PhD. 

The formal proving process is not merely a mechanical implementation process. Previous proofs of the correctness of the Viterbi algorithm were written to satisfy a human reviewer, not to satisfy a formal computer verifier. Therefore, they almost certainly handwave parts of the proof, and have gaps in them which they fail to prove to a sufficient standard of rigor. In order to fill in the gaps in the proof, it will be necessary to come up with new ideas, rather than merely implementing old ones. Thus, the project also has the element of research that is expected of a PhD (rather than being pure development work).

The formalization will require a significant amount of work, because proving something to the requisite standard of rigor necessary for formal proof requires it to be proven to a high level of detail. Furthermore, there is ample room for extending the project in case it takes less work than expected. In 2016, R Affeldt et al said that formal verification of error-correcting codes is understudied [6], and based on my preliminary review of the literature, I agree that there is a significant amount of room for expansion. Thus, even in the case where the production of a formally verified Viterbi algorithm takes less work than expected, it should be possible to extend the project by performing more work, for example, by proving the correctness of other error correcting code algorithms. Therefore, it should be possible to perform sufficient work in this area to be appropriate for a PhD.

On the other hand, the amount of work required can be realistically achieved within a PhD project. Interactive theorem provers are a proven technology that have been used to produce formally verified implementations of several algorithms, for example a formally verified OS kernel [7] and a formally verified compiler of a pure programming language [8]. This shows that the techniques necessary to produce a formally verified implementation of even complex algorithms is available. On the one hand, the formal implementations of these algorithms involved several authors, but on the other hand they also were implementations of relatively complicated systems, so in my estimation the amount of work required to produce a formally verified implementation of the Viterbi algorithm should be appropriate for a PhD.

To evaluate my progress mid-way through the PhD, it should be possible to look at the proofs and specifications that have been produced so far. The project will have several stages which will need to be completed before the project as a whole can be completed. For example, we will need to create a specification for the Viterbi algorithm, we will need to prove theorems describing the mathematical background that the Viterbi algorithm is based upon, and we will need to prove the correctness of individual parts of the Viterbi algorithm. Then in order to evaluate how much progress I have made in my PhD, I should be able to look at how many proofs/specifications have been written so far. This would allow us to have a sense of whether or not progress was being made on the project, providing a greater guarantee that the PhD program would be completed successfully and on time.

In summary, my proposal is to use an interactive theorem prover to produce a formally verified implementation of the Viterbi algorithm. This would be useful because error-correcting codes are necessary in a wide variety of applications, including those in which it is necessary to have a high level of confidence in the correctness of the software. This project would be feasible, as similar projects have been completed before, and this project would take a significant amount of work, as a high level of detail would be necessary to provide a sufficiently rigorous formal proof. It would furthermore be possible to evaluate the progression of the project by evaluating the work that has been completed so far, which could include a specification for the Viterbi algorithm or the proof of sub-theorems necessary in the correctness proof of the overall algorithm.

References:

[1]    J. Baylis, Error Correcting Codes: A Mathematical Introduction. Boca Raton: Chapman \& Hall, 1998.

[2]    A. Viterbi, “Error Bounds for Convolutional Codes and an Asymptotically Optimum Decoding Algorithm,” IEEE Transactions on Information Theory, vol. 13, pp. 260-269, Apr 1967.

[3]    Y. Bertot, P. Casteran, Interactive Theorem Proving and Program Development: Coq’Art: The Calculus of Inductive Constructions. Heidelberg: Springer-Verlag, 2004.

[4]    M. C. Vuran, I. F. Akyildiz, “Error Control in Wireless Sensor Networks: A Cross Layer Analysis”, IEEE/ACM Transactions on Networking, vol. 17, pp. 1186-1199, Aug 2009

[5]    D. Patterson, G. Gibson, R. H. Katz, “A Case for Redundant Arrays of Inexpensive Disks (RAID),” Proceedings of the 1988 ACM SIGMOD International Conference on Management of Data, pp. 109-116, Jun 1988

[6]    R. Affeldt, J. Garrigue and T. Saikawa, “Formalization of Reed-Solomon codes and progress report on formalization of LDPC codes,” International Symposium on Information Theory and Its Applications, pp. 532-536, Oct 2016.

[7]    G. Klein, K. Elphinstone, G. Heiser, J. Andronivk, D. Cock, P. Derrin, D. Elkaduwe, K. Engelhardt, R. Kolanski, M. Norrish, T. Sewell, H. Tuch, S. Winwood, “seL4: Formal Verification of an OS Kernel,” Proceedings of the ACM SIGOPS 22nd symposium on Operating systems principles, pp. 207-220, Oct 2009.

[8]    R. Kumar, M. O. Myreen, M. Norrish, S. Owens, “CakeML: a verified implementation of ML,” ACM Sigplan Notices, vol. 49, pp. 179-191, Jan 2014.

\section{Literature Review}

\subsection{Overview of literature}

This week I skimmed through many papers which I previously identified as being relevant to my research, to identify the main contributions of each of them. I have now skimmed the literature sufficiently thoroughly to be confident that my proposed research area doesn't overlap with any existing research (my previous assessment of the situation had been tentative, based on a more cursory literature review).

I count 11 papers that involve formal verification (FV) related to Error Correcting Codes (ECCs) through means of interactive theorem provers (ITPs), and 8 additional papers on the subject of formal verification of ECCs through other means.

Only one paper has included formalization of convolutional codes. This paper was written in 2020, and the paper itself claims to be the first formalization of a convolutional code. I agree with this assessment, based on my own examination of the literature. This paper used ACL2 for its formalization. My understanding is that this paper did not formalize the Viterbi algorithm, instead focusing on a different convolutional algorithm.

Several papers (including the above paper) have provided libraries of useful functions relating to error-correcting codes. Most notably amongst them is "A library for formalization of linear error-correcting codes". This paper uses the Coq proof-assistant to formalize Hamming, Reed-Solomon, and BCH codes.

The above code is contained in the following git repository: https://github.com/affeldt-aist/infotheo. My understanding is that this repository contains the code pertaining to 6 other papers, 3 of which are directly related to ECCs. These papers formalize Hamming codes, Reed-Solomon codes, BCH codes, cyclic codes, important parts of LDPC codes, and also some theoretical foundations such as Shannon's theorems.

Another pair of papers formalizes the notion of insertion/deletion codes using Lean. These differ from typical ECCs in that, rather than merely having symbols be modified or replaced with a placeholder symbol (erased), they are designed to handle situations in which symbols may be arbitrarily inserted into or deleted from the string entirely, thus changing the length of the string. This is a separate problem to the one I am working on, but nevertheless interesting and unique.

Several papers are less unique. It seems that hamming codes, repetition codes, and BCH codes have been formalized to death already.

I also skimmed a few textbooks, to identify theory that is likely to be relevant to my project. In particular, I found that "A course in error-correcting codes" by J Justesen and T Høholdt contained an explanation of convolutional codes, including the Viterbi algorithm.

I think that "Modern Coding Theory", by T. Richardson and R. Urbanke is also relevant. It contains an explanation of turbo codes, which by my understanding are a newer kind of convolutional code.

I also read a chapter in "Error-Correcting Codes: A Mathematical Introduction" about Linear codes, and added my notes to my thesis proposal review (which is currently in the extremely rough drafting stage, merely containing a bunch of notes that may be useful at some point)

I also spent two full days at the formal methods workshop, during which time I wasn't able to perform other research activities. Notable things I learnt about during the formal methods workshop include the K semantic framework, which can be used to generate formal verification code for a wide variety of theorem provers and programming languages, easing compatibility issues between different theorem provers and languages. I heard about Sel4, which is a formally verified OS microkernal. I heard about Ironbark, a formally verified CPU architecture. I also heard about a few less directly relevant things, for example, methods for ensuring the robustness of neural networks (e.g. ensuring that tiny perturbations in the inputs tends to cause only a tiny perturbation in the output), and even less relevantly, how a person's data record can be de-identified if you know a small amount of personal information about that person.

I generally talked to a lot of academics there, and I'm glad I went, because it is my first experience at a conference-type event, and so I got to experience what a conference is like.

\subsection{Viterbi related}

\subsubsection{Wikipedia}

The Viterbi algorithm is an algorithm which allows you to determine the most likely sequence of states in a hidden Markov model that lead to a certain sequence of observations.

Note that this sequence of states may not necessarily have a high probability, but it is the most likely sequence of states out of all possible sequences of states.

This algorithm is used in decoding convolutional codes. For example, (direct quote here):  ``CDMA and GSM digital cellular, dial-up modems, satellite, deep-space communications, and 802.11 wireless LANs.''

(direct quote) ``It is now also commonly used in speech recognition, speech synthesis, diarization,[1] keyword spotting, computational linguistics, and bioinformatics''

It is a dynamic programming algorithm.

Let $P_{s,t}$ denote the probability of the maximum probability path ending in the state $s$ at the time $t$.

At $t = 0$, we know that $P_{s,t}$ is equal to the prior probability of being in that state multiplied by the probability of seeing the first observation if you are in that state.

At each time beyond that, $P_{s,t}$ is equal to the maximum over all states $s'$ of $P_{s', t-1}$ multiplied by the transition probability between $s'$ and $s$, multiplied by the probability of seeing the observation that was observed at time $t$ if we are in state $s$.

Also make sure to keep track of the transitions that were used to obtain each maximum probability path.

Thus we can use dynamic programming techniques to first calculate all the probabilities at time 0, then calculate all the probabilities at time 1, then calculate all the probabilities at time 2, etc. This will avoid having to re-calculate the probabilities at time 0 multiple times when, for example, finding the maximum probability paths to several different states at a time far in the future. In particular, this will not need to be calculated once for every possible path it is possible to take, but rather, it will only need to be calculated once.

\subsubsection{(high importance) Viterbi's original paper: Error bounds for convolutional codes and an asymptotically optimum decoding algorithm }

\subsubsection{Viterbi algorithm for error correction}

\subsubsection{(unknown quality, google search) The Viterbi Algorithm, by G David Forney, JR}

\subsubsection{Wikipedia Convolutional Codes}

Viterbi is used for small values of $k$. Longer constraint length codes are often decoded using sequential decoding algorithms e.g. the Fano algorithm.

\subsubsection{Turbo codes}

More modern approach than Viterbi that approach theoretical limits imposed by Shannon's theorem.

% --------------------------------------------------------------------------------------------------------------------------------------------------------------------------------

\subsection{Formal ECC Verification}

\subsubsection{A Library for Formalization of Linear Error-Correcting Codes 2020, 8 citations}

Systematic formalization of  linear error correcting codes in the Coq proof-assistent.

Hamming, Reed-Solomon, BCH

\subsubsection{Formalization of Reed-Solomon codes and progress report on formalization of LDPC codes, 2016}

Proof-assistent based formalization of Reed-Solomon codes.

\subsubsection{Formalization of Error-correcting Codes: from Hamming to Modern Coding Theory, 2015, 23 citations}

Coq formalization of Hamming codes adn sum-product algorithm for decoding LDPC codes.

\subsubsection{Verified erasure correction in Coq with MathComp and VST}

Coq formalization of Reed-Solomon coding.

\subsubsection{Formal verification of ECCs for memories using ACL2}

Tail biting

\begin{itemize}
\item The purpose of this work is to design formally verifiable error correcting
codes to correct soft errors in memories

\item Soft errors cause the memory to temporarily return an incorrect bit, due
interference from outside sources. Not a permanent fault, but a temporary one.

\item One of the most notable (but irrelevant to me) features of this work is
that it explicitly involves the handling of memories. In particular it
implements them in a formal record-based memory model.

\item Typically, encoding/decoding of convolutional codes happens over several
clock cycles. However, this implementation uses a combinatorial method.
My understanding is that this increases the rate at which error correction
can occur. Apparently combinatorial circuits are ones which do not use memory.

\item In this implementation, rather than taking a stream of bits as input, it
breaks it down into blocks. Encoding starts in state 00, and two null bits
are placed at the end, in order to provide enough padding to properly
encode all of the bits. This paper refers to this technique as tail-biting,
but I believe that this is incorrect. This technique appears to be
zero-tailed encoding (see the mathworks page on tail-biting convolutional
coding).

\item Mentions that output bits may be interleaved to minimize data corruption.
Not entirely sure what this means. Source is a patent, hard to find
description of what interleaving precisely means.

\item Combinatorial decoder is based on another paper: "Convolutional Coding for SEU
mitigation"
\end{itemize}

\subsubsection{Formalization of coding theory using lean}

Repetition codes, Hamming (7,4) code, basics of coding theory formalized.

\subsubsection{A Coq formalization of information theory and linear error-correcting codes. https://github.com/affeldt-aist/infotheo}
\subsubsection{Formalization of theorems about stopping sets and the decoding performance of LDPC codes}

Cited by "A library for formalization of linear error-correcting codes", but I can't find it on the internet

\subsubsection{Formalization of insertion/deletion codes and the Levenshtein metric in lean}

Lean: fundamental concepts regarding error-correcting codes capable of correcting insertions, deletions, or combinations of insertions and deletions. In particular, we formalize definitions and theorems about subsequences and supersequences, the Levenshtein distance, insertion/deletion spheres, and insertion/deletion codes.

See also: Coding for Insertions and Deletions, quote: in fact, our understanding of insertion-deletion codes significantly lags behind our thorough understanding of error correcting codes (ECCs)


\subsubsection{Formalization of Shannon’s Theorems}

Formalization of Shannon's Theorems in Coq

\subsubsection{A Pragmatic Approach to Extending Provers by Computer Algebra—with Applications to Coding Theory}

Isabelle case study in coding theory. Hamming codes and BCH codes.

\subsubsection{Coding Theory of Permutations/Multipermutations and in the LEAN Theorem Prover}

The first contribution is a set of files that formalize relevant basic mathematical definitions and theorems as well as the famous Repetition codes and Hamming (7,4) code. The second contribution is a set of files that formalize Levenshtein codes and related definitions and theorems

\subsubsection{(unkown quality, google search) Pragmatic Formal Verification of Sequential Error
Detection and Correction Codes (ECCs) used in
Safety-Critical Design (Aman Kumar) 2024}

(Seems relevant, more review necessary)

Formal verification

Mentions Hamming codes, Hsaio codes, Reed-Solomon codes and Bose-Chandhuri-Hocquenghem codes

(direct quote) ``The ECC used in this work is a combinatorial Quad Bit Error Detection, Triple Bit Error Correction (QEDTEC) ECC''.

Prior approach was simulation-based, testing style approach

New approach use mathematical analysis and proof methods.

Keyword property is used in the formal prover

What formal prover is used?

What specific ECC is used? Description is ambiguous

Formal verifier takes design and desired properties as inputs.

(systemverilog assertions)

Seems to use SystemVerilog

For larger designs, can take a long time to verfiy.

Algorithm consists of Syndrome generator, error detection unit and error correction unit.

My algorithm works in a completely different way using completely different tooling. It uses more authomated kinds of proof techniques

\subsubsection{Formal Verification by The Book: Error Detection and Correction Codes 2020}

UVM-based in previous work.

Seems to be in a similar field to Aman Kumar's work, and not related to interactive theorem provers. Therefore not relevant to my work.

\subsubsection{Verification of an error correcting code by abstract interpretation, by Charles Hymans 2005}

Validated Reed-Solomon

VHDL (VHSIC Hardware Description Language)

Also uses formally verified programming, not interactive theorem provers.

Uses abstract interpretation: unrelated to my work.

\subsubsection{Formal verification of error correcting circuits using computational algebraic geometry, 2012}

Purpose built tool, for error correcting circuits. (BLUEVERI)

Uses VHDL

\subsubsection{Verification of Galois field based circuits by formal reasoning based on computational algebraic geometry}

BLUEVERI-based, programming language based.

\subsubsection{Error correction code algorithm and implementation verification using symbolic representations}

 Triple Error Correction Quadruple Error Detection (TECQED), etc

\subsubsection{UVM-based verification of ECC module for flash memories}

SystemVerilog, UVM

\subsubsection{Formal verification of sequential galois field arithmetic circuits using algebraic geometry}

VLSI, RATO, seems unrelated.

% --------------------------------------------------------------------------------------------------------------------------------------------------------------------------------

\subsection{Error-Correcting codes theory}

\subsubsection{Error correcting codes: A  mathematical introduction (1998)}

\paragraph{Linear codes}

0 is in all linear codes

linear code minimum distance is given by minimum weight

linear codes correction depends only on error vector, not on  specific code

For each position in a linear code, it is either always 0, or every option for this position appears equally often.

linear codes agree with basic linear algebra

Generator matrix: Take a basis for the code space, and create a matrix $G$ in which each row is an element of the basis. Then for each message m, the codeword is described by $m G$

Particularly convenient if the generator matrix has the identity if you cut off the last few columns.

Hamming codes are linear, with identity at beginning.

Consider the codes in the codespace (a.k.a. those generated by the generator matrix). Write these down with 0 at the beginning. Pick a code not in the codespace with the lowest possible weight. Consider the coset by adding this code to each of the codes in the codespace. Pick another code, which is neither in the codespace nor the new coset, with the lowest possible weight, and consider its coset. Continue this process until each code is in some coset. Write these cosets in an array, where each row contains a coset (including the coset where nothing is added to the codespace), and the first element in the first column is 0, and elements in the same column are derived by adding the corresponding code to the original code. The result is called the "Slepian array"

You can decode a word using the Slepian array by finding it in the array, and returning the corresponding element in the first row. This method is equivalent to nearest neighbour decoding.

The coset leader is the number which is added to the original element of the codespace to create the coset. They are the elements in the first column, because 0 + anything is the original thing.

Although different elements can be chosen to generate the cosets, the set of cosets will always be the same if ordering is ignored. That is, the cosets can be swapped around, and elements inside the cosets can be swapped around, but corresponding cosets will contain corresponding elements, and we will always have corresponding cosets no matter how we choose the elements to generate them.

Parity check matrix: independent rows, codespace is the null space of the matrirx

Every linear code has a parity check matrix.

Dual code: If $C$ is an $[n,k]$ code over $Z_p$, then $C^\perp$ is an $[n, n - k]$ code

Matrix is parity check matrix for $C$ if and only if it is a generator matrix for $C^\perp$

Note: an $[n,k]$ code has $n$ dimensions in the encoded form and $k$ dimensions in the decoded form.

Syndrome decoding: allows you to avoid storing whole Slepian array for decoding, and avoid having to search through the whole array to find the element to decode.

Let $H$ be parity check matrix, $v$ any word. Then syndrome of $v$ is $vH^T$

Two elements are in the same coset iff they have the same syndrome.

If $C$ is $[n,k]$ code and $H$ is any parity check matrix for it, $e$ is error pattern associated with received word $r$, then syndrome of $r$ is $\left(\sum\limits_{i=1}^n e_i h_i\right)^T$, where $h_i$ is the $i$th column of $H$.

Special case for binary codes: look into more.

In syndrome decoding, just need coset leaders, syndromes, and parity check matrix

Minimum distance of linear code $C$ is the size of the smallest dependent set of columns of $H$.

\paragraph{Hamming Family}

Hamming codes: Ham$(r, q)$ is the set of those linear $[n, k]$ codes whose parity check matrices have $r$ rows and $n$ columns, where $n$ is the largest possible number of columns that have no pair of columns dependent.

All codes in Ham$(r, q)$ have length $n = \frac{q^r - 1}{q - 1}$

All codes in Ham$(r,q)$ are linearly equivalent to each other.

All Hamming codes have a minimum distance of 3

All Hamming codes are perfect

(todo what does this mean) In any linear code, the distribution of codeword weights is indentical to the distance distribution.

The dual of any Hamming code is called a simplex code. 	

All simplex codes are equideistant codes.

todo: continue from theorem 6.7

\subsubsection{(found via google search) A course in error-correcting codes, by J Justesen, T Høholdt (2004)}

Convolutional codes explanation seems too theoretical and not explained in an intuitive way

Includes convolutional codes info

Decoding Reed Solomon

Decoding BCH

Algebraic geometry

\paragraph{Convolutional codes}

direct quote: "The most important limitation on algebraic decoding algorithms is that only t < d=2 errors can be corrected, while in most cases error patterns of significantly higher weight could in principle be corrected"

Generally correcting codes of highest probability, instead of making sure to correctly decode all codes with less than a certain number of errors. Mathematically most likely message.

Better noise handling? (uncited)

\subsubsection{(cited by Aman Kumar, book) Error Detection and Correction Codes by Diego L. Gonzalez (2008)}

Difficult to get a hold of, as our library doesn't contain the book it is contained in. Book is from a biological perspective, although the 

Moderate relevance. Information on ECCs

\subsubsection{(found google search different book) A Course in Algebraic Error-Correcting Codes (2020)}

Doesn't seem to include convolutional stuff

Shannon's Theorem, Finite Fields, Block Codes, Linear Codes, Cyclic codes, Maximum Distance Separable Codes, Alternant and Algebraic Geometric Codes, Low Density Parity Check Codes, Reed-Muller and Kerdock Codes, p-Adic codes

\subsubsection{The Theory of Information and Coding, Encyclopedia of Mathematics and its Applica-
tions}

A little bit old. Only 4 sections on convolutional codes

Information theory: entropy, discrete memoryless channels, discrete memoryless sources, the gaussian channel, the source-channel coding theorem

Coding theory: linear codes, cyclic codes, BCH, Reed-Solomon, convolutional codes, variable-length source coding

\paragraph{Introduction}

Binary symmetric source: outputs bits at random, 1 equally likely as 0, R symbols per unit time. (For example, if R = 1/3, then 3 bits can be transmitted over the channel for every bit that is produced by the source. With rates larger than 1, we can transmit some symbols and require the reciever to guess the remaining ones.)

Binary symmetric channel: transmit 1 bit per unit time, probability p that the output bit will flip.

Bit error probability: expected number of bit errors per unit time

Discrete memory

(n, k) codes. These codes have rate $R = k/n$

Hamming codes.

Equation for the boundary between achievable $(R, P_e)$ pairs: $R = \frac{1 - H_2(p)}{1 - H_2(P_e)}$, for $p$ the probability of bit error in the channel, $R$ the rate of the code, $P_e$ the bit error probability of the code, and $H_2$ the binary entropy function, given by $H_2(x) = -x\log_2 x - (1 - x) \log_2(1-x)$.

If $R < 1 - H_2(p)$, then arbrarily low $P_e$ is achievable! That is, there is a bit rate at which we can have arbitrarily few errors.

\paragraph{Channel coding Theorem}

Essential idea is that for any discrete memoryless channel, there is a channel capacity $C$ for which  it is possible to find algorithms which approach $C$ in rate and approach 0 in bit error probability


\subsubsection{Modern Coding Theory. (probably intend the one by T. Richardson, R. Urbanke, 2008)}

Factor graphs, binary erasure channel, binary memoryless symmetric channels, general channels, turbo codes, general ensembles, expander codes and flipping algorithm,

\subsubsection{Robust error detection in communication and computational channels, 2007, 72 citations}

\subsubsection{MIT Convolutional coding lecture notes (2010)}

\paragraph{Lecture 4}

Noise can be created by random path taken by electrons, the effect of many components unpredictably drawing power from the same source, IR tranceivers can have noise from room light.

If this causes a change beyond a threshold, it may cause bit errors.

\paragraph{Lecture 5}

Eye diagram? PDFs of noise + detection V's? Deconvolution.

\paragraph{Lecture 6}

Information about linear, Hamming, SEC

\paragraph{Lecture 7}

Adler-32 checksum. Cyclic redudancy check

\paragraph{Lecture 8}

\begin{itemize}
\item A sliding window is moved over the input (this is why the code is called convolutional). At each position a number of parity bits are generated. These parity bits comprise the output.

\item The constraint length K denotes the size of the sliding window

\item If r bits are generated at each position, then the output will be r times
the size of the original, and the code rate will be 1/r, as 1 bit of the
original message is transmitted per r bits of the encoded message.

\item In a given window, the parity bits are generated using partity equations,
which are polynomials over the elements of the window, in modulo 2.

\item Convolutional coding can be viewed in terms of a block diagram, which
effectively interprets the algorithm in the aforementioned way, as outputs
generated from polynomials

\item It can also be viewed from a state machine perspective, where each state
represents the past few perceived bits, and each transition refers to the
bit that was read in this state and the bits that were output in this
state, as the polynomials will always output the same bit if they are in
the same state.

\item Typically starts from the state which perceived two zeros beforehand

\item The above is sufficient for encoding

\item Decoding through maximum likelihood decoding. This is equivalent to
finding the nearest valid codeword in terms of Hamming distance.

\item The trellis is a structure that contains the state at each time step,
and transitions from states at each time step to the corresponding states
at the next time step. Basically adds time information to the state
machine.

\item Hard decision decoding is basically digital whereas soft decision decoding
is basically analog. This algorithm is typically implemented on the
circuit level, where we have voltages rather than binary 1s and 0s. Soft
decoding improves error correcting performance.

\item Want to find a path through the trellis that most closely approximates the
received message.

\item Branch metric: Hamming distance between output of a particular transition
and the received message at that point.

\item Path metric: the minimum sum of branch metrics on any path in order to get
to a particular state.

\item The Viterbi algorithm basically just applies dynamic programming to the
trellis in order to minimize the path metric at each state and time. The
path metric of any state at a given time is dependent on and only on the
path metrics of all states at the previous point in time.

\item Downside: decoding time exponential in K, the constraint length

\item Downside: decoding time for first bit requires the entire path to be built

\item In practice, starting decoding at 5 * K bits is a reasonable decoding
window, because beyond this point it's unlikely for further future
knowledge to affect the optimal first move on the path

\item My own observation: surely at some point, the optimal first move will be
optimal on a path towards each state, and at that point you could take
that step confidently?

\item BCJR or Fano's sequential decoding scheme may be used to decode for large
K.
\end{itemize}

\subsubsection{Convolutional coding for SEU mitigation (cited by ACL2 paper)}

\begin{itemize}
\item Discusses ECCs as a method for correcting soft errors

\item In this case, tail-biting seems to refer to a circular data word, where
the last bit wraps around to the first one.

\item The problem under consideration requries fast algorithms, so typical
stream-based methods are not appropriate. This is why the combinatorial
method for convolutional encoding/decoding was developed.

\item To minimize complexity and improve speed, focuses on single-bit
correction.

\item Very specific choice of generating polynomials: (1, 1, 1) and (1, 0, 1)

\item Unfolded over entire word in order to improve speed, transforming from
sequential into combinatorial

\item Uses custom, seemingly novel, combinatorial method for decoding.

\item Four received bits can uniquely identify the two bits which must have
generated them. Thus, easy to decode uniquely in the case where we know
that no errors have occurred.

\item Firstly, decode uniquely as if there were no errors, and overlap the
decoding modules to create another decoding option as if there were no
errors.

\item From this pair of decoding options, it is possible to reconstruct the
decoded bit under the assumption of only a single error.

\item Can correct one error per 5 bits, if well spaced.

\item In first stage, decodes uniquely as if there were no errors
In second stage,
\end{itemize}

\subsubsection{Fundamentals of Convolutional Coding (1999 original, 2015 current)}

Reed solomon codes are particularly good at dealing with burst errors

Viterbi good for white Gaussian noise.

Viterbi can easily exploit soft-decision information, unlike Reed-Solomon.

Concatenated coding system: Viterbi decoder corrects most channel errors but occassionally outputs a burst of errors. Reed-Solomon is well adapted to cope with burst of errors.

My own observation: perhaps the reason that convolutional codes work less well for burst errors is that by design, they focus on nearby bits, whereas block codes by design include bits far away? Similarly, convolutional codes work best for errors that are regularly spaced.

\subsubsection{Hamming, R.W.: Error detecting and error correcting codes.}

Hamming's original paper

\subsubsection{Viterbi A: Convolutional Codes and their Performance in
Communication Systems (1971)}

Viterbi's original paper? Or maybe one of their papers

\subsubsection{Polynomial codes over certain finite fields}

Original Reed-Solomon paper

\subsubsection{Wikipedia}

\paragraph{Convolutional code} Turbo codes and LDPC codes are more modern, coming arbitrarily close to the Shannon limit.

Errors appear in bursts, which must be accounted for. This is typically accounted for by interleaving data before convolutional coding, so that the outer code (typically Reed-Solomon) can correct the majority of the errors

Typically predefined codes based on research papers are used in the industry, to avoid accidentally selecting a convolutional code with bad properties.

Used in Intelsat and Digital Video Broadcasting

Turbo codes are affected by an error floor: concatenation with an algebraic code helps solve this issue.

\paragraph {Error correction code}

By interleaving, the errors are spread out, making errors more uniformly distributed, avoiding issues caused by bursts of errors.

downside: may increase delay because entire interleaved packet must be received before it can be decoded.

\paragraph{Capacity-approaching codes}

\paragraph{Error floor}

The error floor is a phenomenon encountered in modern iterated sparse graph-based error correcting codes like LDPC codes and turbo codes. When the bit error ratio (BER) is plotted for conventional codes like Reed–Solomon codes under algebraic decoding or for convolutional codes under Viterbi decoding, the BER steadily decreases in the form of a curve as the SNR condition becomes better. For LDPC codes and turbo codes there is a point after which the curve does not fall as quickly as before, in other words, there is a region in which performance flattens. This region is called the error floor region. The region just before the sudden drop in performance is called the waterfall region.[1]

Error floors are usually attributed to low-weight codewords (in the case of Turbo codes) and trapping sets or near-codewords (in the case of LDPC codes).[2]

\paragraph{Low-density parity check}

Affected by error floor, sometimes adds Reed-Solomon on top

5G uses polar code for control channels and LDPC for data channels.

Although LDPC code has had its success in commercial hard disk drives, to fully exploit its error correction capability in SSDs demands unconventional fine-grained flash memory sensing, leading to an increased memory read latency. LDPC-in-SSD[22] is an effective approach to deploy LDPC in SSD with a very small latency increase, which turns LDPC in SSD into a reality. Since then, LDPC has been widely adopted in commercial SSDs in both customer-grades and enterprise-grades by major storage venders. Many TLC (and later) SSDs are using LDPC codes. A fast hard-decode (binary erasure) is first attempted, which can fall back into the slower but more powerful soft decoding

\subsubsection{Other}

One advantage of convolutional codes is that they allow for soft-decision encoding 

Polar codes “Polar Code Appropriateness for Ultra-Reliable and Low-Latency Use Cases of 5G Systems”

Canberra Deep Space Communication Complex for the Deep Space Network

\subsubsection{Non-reliable}

Non reliable source: Apparently convolutional codes are better for burst errors whereas block codes are better for more evenly distributed errors?

Non reliable source: apparently hardware is more complex for block codes than for convolutional codes?

\subsubsection{Books to consider}

A Student's Guide to Coding and Information Theory

INFORMATION THEORY AND CODING: A Fundamental Approach

Information Theory and Coding
by Murlidhar Kulkarni and K.S. 

Coding and Information Theory: 134
by Steven Roman

Coding Theory: The Essentials: 150
by D. G. Hoffman, D. A. Leonard, et al.

Information Theory and Coding: Information, Source Coding and Channel Coding
by Dr. J. S. Chitode

Coding and Information Theory
by Richard W. Hanning

Information Theory, Coding and Cryptography

Information-Spectrum Methods in Information Theory: 50
by Te Sun Han and H. Koga
5.0 out of 5 stars

Convolutional Coding: Fundamentals and Applications
by L H Charles Lee and Charles Lee

% --------------------------------------------------------------------------------------------------------------------------------------------------------------------------------

\subsection{Other theory}

Soft errors are non-repeatable, temporary errors caused by external interference.

Hard errors are caused by a repeatable problem, e.g. a problem in the circuit or algorithm.

% --------------------------------------------------------------------------------------------------------------------------------------------------------------------------------

\subsection{Other formal verification}

\subsubsection{Verification, Model Checking, and Abstract Interpretation conference}

\subsubsection{Interactive Theorem Proving conference}

\subsubsection{(cited by Charles Hymans) A static analyzer for large safety-critical software}


\subsubsection{Vale: Verifying {High-Performance} Cryptographic Assembly Code}

\subsubsection{(cited by Aman Kumar) Formal Verification, An Essential Toolkit for Modern VLSI Design} 

\subsubsection{Formalizing 100 theorems}

\subsubsection{Bel-Games: A Formal Theory of Games of Incomplete Information Based on Belief Functions in the Coq Proof Assistant}

\subsubsection{Reasoning with Conditional Probabilities and Joint Distributions in Coq}

\subsubsection{An Introduction to MathComp-Analysis}

\subsubsection{Formalization of Shannon’s theorems}

\subsubsection{Solutions to IBM POWER8 verification challenges}

\subsubsection{Verification of Composite Galois Field Multipliers over GF (2\^m\^n) Using Computer Algebra Techniques}

\subsubsection{Examples of formal proofs about data compression}

\subsubsection{Classification of finite fields with applications}

\subsubsection{Certifying assembly with formal security proofs: the case of BBS}


% --------------------------------------------------------------------------------------------------------------------------------------------------------------------------------

\subsection{Soft errors (what are these? Probably unrelated)}

\subsubsection{(cited by Aman Kumar, book) Soft errors in Modern Electronic Systems}

\subsubsection{(cited by Aman Kumar, book) Dependability in electronic systems - Mitigation of Hardware Failures, Soft Errors, and Electro-Magnetic Disturbances}

Probably only tangentially relevant

% --------------------------------------------------------------------------------------------------------------------------------------------------------------------------------

\subsection{Low relevance}

\subsubsection{(cited by Aman Kumar) Error correcting code analysis for cache memory high reliability and performance}

Low relevance


\subsection{(cited by Aman Kumar) Formal verification by the book: Error detection and correction codes}
Signals and Communication Technology
Martin Tomlinson
Cen Jung Tjhai
Marcel A. Ambroze
Mohammed Ahmed
Mubarak Jibril
Moderate/High relevance

\subsection {(from google) Error-Correction Coding and Decoding by Martin Tomlinson, Cen Jung Tjhai, Marcel A. Ambroze, Mohammed Ahmed, Mubarak Jibril}

Lots of interesting codes, e.g. algebraic geometry codes

\subsection {Joe Hurd PhD thesis Formal verification of probabilistic
algorithms (recommended by Michael)}

The thesis is aqbout how to fomally verify probabilitic algorithms. Not specific to HOL4, although HOL4 is the theorem prover that was used.

\begin{itemize}
\item {Formally verified probabilistic algorithms example}
\item {Probability foundations}
\item {Probabilistic program modelling}
\item {Verify Miller-Rabin test}
\item {}
\item {}
\item {}
\item {}
\end{itemize}


\subsection{Formalizing Integration Theory, with an
Application to Probabilistic Algorithms
Stefan Richter 
(unkown quality, found via google because it cites Joe Hurd's thesis)}

Conclusion: unlikely to be relevant. In Isabelle. On the other hand, is an approach for formalization of Probability-relevant background.

\begin{itemize}
\item {Uses Isabelle}
\item {Lesbegue integration}
\item {Sigma algebra}
\item {Monotone convergence}
\item {Measure spaces}
\item {Real-valued random variables}
\item {Integration}
\item{Probability spaces}
\end{itemize}

\section{Other notes}

\subsection{Conference vs Journals}

Conference typically has faster turnaround than journals do (Michael)

\subsection{Peer Review in Computer Science}

Peer review is relatively low in computer science compared to other fields and especially low in workshops (Michael). Proceedings often available well before actual conference.

E.g. arXiv pre-prints.

\section{Research activities since commencement}

\subsection {Fermat's Little Theorem}

\subsection {Simple block codes proof}

\subsubsection {Block codes theory}

\subsubsection{}	


\subsection {Extreal to real, polynomial solver.}

\subsection{Literature Review}

\subsubsection{Identification of Relevant Literature}

\subsubsection{Review to identify main contributions of relevant literature}

\section{Methodology to be Employed}

The HOL4 interactive theorem prover will be used to prove theorems.

\section{How my research will contribute to my field of study}

\section{Bibliography}


\end{document}
